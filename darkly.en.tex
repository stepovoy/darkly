\section{Introduction}

When you develop your first websites, you usually do not have any idea
of the existing vulnerabilities in the world of the web. This small
project aims to fill this gap: you will become aware of these
vulnerabilities by doing an audit of a simple website. This site has
flaws still regularly present on sites that you visit every day. Here is
a big introduction to the general vulnerabilities found in the world of
the web.

\section{Objectives}

This project aims to introduce you to computer security in the field of
the web. You will be able to discover OWASP, which is, neither more nor
less, the biggest project of web security to date. You will also
understand what many frameworks do completely transparent for you.

\section{General Instructions}

\begin{itemize}
\tightlist{}
\item
  This project will only be corrected by humans.
\item
  During your defense, you may be required to prove your results. You
  must prepare for it.
\item
  You need to use virtual machine (i386) to validate this project. Once
  your machine launched with the ISO provided with the subject, if
  everything is well configured, you will have a simple prompt with an
  IP\:
\end{itemize}

\begin{verbatim}
 ____                _______    _____
|  _ \              |__   __|  / ____|
| |_) | ___  _ __ _ __ | | ___| (___   ___  ___
|  _ < / _ \| '__| '_ \| |/ _ \\___ \ / _ \/ __|
| |_) | (_) | |  | | | | | (_) |___) |  __/ (__
|____/ \___/|_|  |_| |_|_|\___/_____/ \___|\___|
                 WEB SECTION
             Good luck & Have fun
To start the challenges, open your web browser (:80) and go to:
                172.16.60.128
BornToSecWeb login: _
\end{verbatim}

\begin{itemize}
\tightlist{}
\item
  You only need to connect with your browser to ip address displayed.
\item
  Please inform the pedagogical team if you find a bug!
\item
  You can ask your questions on the forum, on jabber, IRC, slack
  \ldots{}
\end{itemize}

\section{Mandatory Part}

\begin{itemize}
\tightlist{}
\item
  Your turn-in folder should only contain the things that allowed you to
  solve each exploited flaw.
\item
  Yout turn-in folder should have the following structure:
\end{itemize}

\begin{verbatim}
$> ls -al
[..]
drwxr-xr-x  2 root root 4096 Dec  3 XX:XX {Nom de faille}
[..]
$> ls -alR {Nom de faille}
{Nom de faille}:
total 16
drwxr-xr-x 3 root root 4096 Dec  3 15:22 .
drwxr-xr-x 6 root root 4096 Dec  3 15:20 ..
-rw-r--r-- 1 root root    5 Dec  3 15:22 flag
drwxr-xr-x 2 root root 4096 Dec  3 15:22 Ressources
{Nom de faille}/Ressources:
total 8
drwxr-xr-x 2 root root 4096 Dec  3 15:22 .
drwxr-xr-x 3 root root 4096 Dec  3 15:22 ..
-rw-r--r-- 1 root root    0 Dec  3 15:22 whatever.wahtever
$> cat {Nom de faille}/flag | cat -e
XXXXXXXXXXXXXXXXXXXXXXXXXXXX$
$>
\end{verbatim}

Where \texttt{Nom\ de\ faille} is name of the fault

\begin{itemize}
\tightlist{}
\item
  In the Resources folder you will put everything you need for prove
  your resolution in defense
\end{itemize}

\begin{quote}
ATTENTION\: Everything in this folder must be able to be explained
clearly without any hesitation. NO binary should be present in this
file.
\end{quote}

\begin{itemize}
\tightlist{}
\item
  If you need to use a specific file present on the ISO of the project,
  you must download it in defense. You must not under any circumstances
  put this one in your repository.
\item
  In the case of using specific external software, you must prepare a
  specific environment (VM, docker, Vagrant).
\item
  As mandatory part, you must exploit 14 different faults.
\item
  During your defense, in some cases, you will be asked for possible fix
  for the flaws you exploited. It is strongly advised to understand
  everything you operate.
\item
  Knowing how to explain is often more important than exploitation
  itself: take the time to understand, and especially to make sure you
  can be understood clearly.
\end{itemize}

\begin{quote}
For the clever (or not) \ldots{} Of course you do not have the right to
use scripts like sqlmap in order to make the exploitation trivial. You
must in any case clearly explain your approach during your defense.
\end{quote}

\section{Bonus Part}

\begin{quote}
Bonuses will only be counted if your mandatory part is PERFECT. By
PERFECT, we obviously mean that it is fully realized, and it is not
possible to alter its behavior in default, even in case of error,
misuse, etc \ldots{} Concretely, this means that if your mandatory part
is not validated, your bonuses will be fully IGNORED.
\end{quote}

As bonus part, you simply need to provide advanced explanations for the
most recognized flaws that you have encountered.

\section{Correction and peer-evaluation}

Make your work on your GiT repository as usual. Only the present work on
your repository will be assessed in defense.
